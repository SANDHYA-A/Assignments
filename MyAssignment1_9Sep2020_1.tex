\documentclass[conference]{IEEEtran}
\IEEEoverridecommandlockouts
\usepackage{cite}
\usepackage{amsmath,amssymb,amsfonts,bm}
\usepackage{algorithmic}
\usepackage{graphicx}
\usepackage{textcomp}
\usepackage{xcolor}
\usepackage{verbatim} 
\newcommand{\myvec}[1]{\ensuremath{\begin{pmatrix}#1\end{pmatrix}}}
\newcommand{\norm}[1]{\left\lVert#1\right\rVert}
    
\begin{document}

\title{Matrix Theory EE5609 - Assignment 1\\ 
}

\author{\IEEEauthorblockN{Sandhya Addetla}
\IEEEauthorblockA{PhD Artificial Inteligence Department} \\
09-Sep-2020\\
AI20RESCH14001\\
 }

\maketitle

\begin{abstract}
This document provides a solution for the problem of finding slopes of two lines,  slope of one line being double of the slope of another line. and tangent of the angle between them is 1/3.
\end{abstract}

\section{Problem Statement}
The slope of a line is double of the slope of another line. If the tangent of the angle between them is 1/3 , find the slopes of the lines.

\section{Solution}
Two vectors are considered to be passing through origin. Where  $m_{1}$   and  $ m_{2}$ are slopes of the vectors $\vec{m_{1}}$   and    $\vec{m_{2}}$ respectively.
Let  $m_{1} = m$. Given  $m_{2} = 2m$

These vectors can be represented as below:-
\begin{align}
\vec{m_{1}} =\myvec{ 1 \\m }
\vec{m_{2}} =\myvec{1\\2m}
 \end{align}
The dot product of the vectors is given by:-
\begin{align}
 \vec{m_{1}} ^T  \vec{m_{2}} = \norm{  \vec{ m_{1}}}\norm{  \vec{m_{2}}}  \cos \theta
\end{align}

Given that $tan \theta$ is $\frac{1}{3}$. By Pythagorus theorem, we can obtain $cos \theta$ as $\frac{3}{\sqrt{10}}$.
Therefore,
\begin{align}
\cos \theta &= \frac{\vec{m_{1}}^T  \vec{m_{2}}}{\vec{\norm{ m_{1}}} \norm{ \vec{m_{2}}}}\\
\frac{3}{\sqrt{10}} &=  \frac{1 \times 1 + {m \times 2m}}{{\sqrt{1 + m^2}\sqrt{1 +4 m^2}}}
\end{align}
Applying square on both sides:-

\begin{align}
9 \times (1 + m^2) (1 + 4 m^2) & = 10 ( 1 + 2 m^2)^2\\ 
4 m^4 - 5 m^2 + 1 &=0\\ 
m_{1} = m &= 1, -1,\frac{1}{2},\frac{-1}{2}
\end{align}

Substituting the value of $ m_{1}$ we get value of\\
$ m_{2} = 2, -2, 1, -1$.\\

\section{Conclusion}
The slopes $ m_{1}$ and $ m_{2} $ of vectors \textit{$\vec{m_{1}}$  and    $\vec{m_{2}}$} for the said conditions are: -
 \begin{center}
\myvec{1 \\ 2} , \myvec{-1 \\ -2} ,\myvec{ \frac{1}{2}\\1} ,\myvec{- \frac{1}{2}\\ -1}
\end{center}

\end{document}

